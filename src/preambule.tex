\documentclass[russian,utf8,pointsection]{eskdtext}
\usepackage{cmap}			% Поиск по русским словам в конечном pdf документе
\usepackage{eskdchngsheet}
\usepackage[T2A]{fontenc}
\usepackage{pscyr}			% Подключение "красивых" шрифтов киррилицы
\usepackage{amstext}
\usepackage{amsmath}
\usepackage{listings}
\usepackage{tablefootnote}

\usepackage{listings}		
\lstset{ %
	language=tcl,           % Язык программирования 	
	frame=single,           % Добавить рамку
	breaklines=true,        % Автоматический перенос строк
	breakatwhitespace=true, % Переносить строки по словам
	title=\lstname 
}

% нумерация таблиц и рисунков по разделам
\usepackage{chngcntr}
\counterwithin{figure}{section}
\counterwithin{table}{section}


% работа с сылками
\usepackage{hyperref}
\usepackage[usenames,dvipsnames,svgnames,table,rgb]{xcolor}
\hypersetup{			
	unicode=true,           							% русские буквы в раздела PDF
	pdftitle={Руководство по эксплуатации (часть 2)},	% Заголовок
	pdfauthor={Щеблыкин М.В.},      					% Автор
	pdfsubject={Интерфейс <<Человек-машина>>},			% Тема
% 	pdfcreator={Latex},			 						% Создатель
% 	pdfproducer={Производитель}, 						% Производитель
% 	pdfkeywords={keyword1} {key2} {key3}, 	% Ключевые слова
	colorlinks=true,      					% false: ссылки в рамках; true: цветные ссылки
	linkcolor=NavyBlue,        				% внутренние ссылки
	citecolor=black,        				% на библиографию
	filecolor=black,        				% на файлы
	urlcolor=black          				% на URL
}

% Дает доступ к командам:
% \MakeTextUppercase{} - сделать все символы заглавными
\usepackage{textcase} 

%Изменение отображения Содержания
%\makeatletter
%\renewcommand{\l@section}{\@dottedtocline{1}{0em}{1.25em}}
%\renewcommand{\l@subsection}{\@dottedtocline{2}{1.25em}{1.75em}}
%\renewcommand{\l@subsubsection}{\@dottedtocline{3}{2.75em}{2.6em}}
%\makeatother

% Работа с таблицами
% p{} - top align, m{} - middle align, b{} - bottom align
\usepackage{ltablex} 										% longtable с функциональностю tabularx
\usepackage{multirow} 										% Слияние строк в таблице
\renewcommand{\tabularxcolumn}[1]{>{\arraybackslash}m{#1}}	% выравнивание в ячейке таблицы по середине по вертикали
\newcolumntype{M}[1]{>{\centering \arraybackslash}m{#1}} 	% колонка с заданной шириной и выравниванием по центру
\newcolumntype{Z}{>{\centering \arraybackslash} X} 			% колонка с выравниванием по центру
\newcolumntype{L}{>{\raggedright\arraybackslash} X} 		% for ragged-right material
\newcolumntype{C}{>{\centering\arraybackslash} X} 			% for centered material

% Добавлено отображение Города на Титульном листе
\renewcommand{\ESKDtheTitleFieldX}{Екатеринбург \\ \ESKDtheYear}

% Уменьшен размер шрифта для заголовков секций
\ESKDsectStyle{section}{\large \bfseries \MakeTextUppercase}

% Выравнивание по центру.
% Предназначено для выравнивания надписей в шапке таблицы
\newcommand{\calign}[1]{\centering #1 \arraybackslash} 

%%% Работа с картинками
\usepackage{graphicx}  			% Для вставки рисунков
\graphicspath{{images/}}  		% папки с картинками
\setlength\fboxsep{3pt} 		% Отступ рамки \fbox{} от рисунка
\setlength\fboxrule{1pt} 		% Толщина линий рамки \fbox{}
\usepackage{wrapfig} 			% Обтекание рисунков и таблиц текстом
\usepackage{float}

%%% Информация
\newcommand{\info}[1]{ %
\begin{flushleft}%
	\begin{tabular*}{\linewidth}{m{0.05\linewidth}m{0.9\linewidth}}%
		\includegraphics[width=\linewidth]{info.jpg} & \textcolor{ForestGreen}{\textit{#1}}
	\end{tabular*}	
\end{flushleft}	
}

%%% Предупреждение
\newcommand{\warning}[1]{ %
\begin{flushleft}%
	\begin{tabular*}{\linewidth}{m{0.05\linewidth}m{0.9\linewidth}}%
		\includegraphics[width=\linewidth]{warning.png} & \textcolor{Red}{\textit{\textbf{#1}}}
	\end{tabular*}	
\end{flushleft}	
}

%%% Переопределение отображения счетчиков enumerate на отображение цифрами
\renewcommand{\theenumi}{\arabic{enumi}}
\renewcommand{\labelenumi}{\theenumi)}

\usepackage[backend=biber,bibencoding=utf8,sorting=ynt,maxcitenames=2,style=authoryear]{biblatex}
\addbibresource{bib_orcad.bib}

%%% используется для рисования
\usepackage{tikz}
\usetikzlibrary{trees}

%\usepackage{ragged2e}

%%% Используется для изменения направления текста
\usepackage{rotating}

\usepackage{caption}

%%% Без куска ниже проект не собирается
\providecommand{\No}{\textnumero}
\makeatletter
\newsavebox\ESKDpicturebox
\renewcommand{\ESKD@ShipoutPicture}{%
     \ifESKD@twoside
       \ifodd\c@page
         \ESKDframeX=\ESKD@margin@si
       \else
         \ESKDframeX=\ESKD@margin@so
       \fi
     \else
       \ESKDframeX=\ESKD@margin@si
     \fi
     \ESKDframeY=\ESKD@margin@b
     \ESKDstampX=\ESKDframeX
     \advance\ESKDstampX \ESKDframeW
     \advance\ESKDstampX -185mm
     \ESKDstampY=\ESKDframeY    
     \sbox\ESKDpicturebox{%
        \unitlength=1mm
        \begin{picture}(0,0)(\ESKDltu{\ESKD@origin@x},\ESKDltu{\ESKD@origin@y})%
          \ifx\ESKD@thisstyle\@empty
            \let\ESKD@thisstyle\ESKD@curstyle
          \fi
          \loop
          \ifnum \ESKD@hash@pos{@style@draw@\ESKD@thisstyle} %
            < \ESKD@hash@count{@style@draw@\ESKD@thisstyle}
            \ESKD@hash@next@value{@style@draw@\ESKD@thisstyle}\relax
          \repeat
          \ifx\ESKD@extra@ThisHook\@empty%
            \ESKD@extra@Hook\else\ESKD@extra@ThisHook%
          \fi%
          \global\let\ESKD@thisstyle\@empty%
          \global\let\ESKD@extra@ThisHook\@empty%
        \end{picture}
        }%
       \AddToHook{shipout/foreground}{%
       \put(1in,-1in){\usebox\ESKDpicturebox}}%        
}

\makeatother
