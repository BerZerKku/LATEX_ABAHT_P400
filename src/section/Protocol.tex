%%% ----------
\subsection{Протокол}	\label{ssec:protocol}

Переход к пункту меню <<Протокол>> из уровня~0 меню: 

\textbf{[*]} $\rightarrow$ \textbf{[7]}.

Внешний вид индикатора данного пункта меню показан на рисунке~\ref{fig:protocol}.
 
 \begin{figure}[H]
 	\centering
 	
	\begin{tabular}{| m{2.5cm}  m{2.5cm} |}
		\firsthline
		330кГц-1	& \raggedleft I1=167мА			\tabularnewline 
		\multicolumn{2}{|l|} {Протокол}				\tabularnewline
		\multicolumn{2}{|l|} {Значение: Стандарт} 	\tabularnewline 
		\multicolumn{2}{|l|} {}						\tabularnewline 
		\lasthline
	\end{tabular} 
	
	\caption{<<Протокол>>}
	\label{fig:protocol}
\end{figure}

Этот пункт меню позволяет выбирать протокол используемый для связи по интерфейсу RS-232: <<Стандартный>>, предназначен для работы с программой <<АВАНТ-конфигуратор>>, или ModBus.

При вводе значений параметров следует руководствоваться общими правилами ввода данных с клавиатуры (см. пункт~\ref{sssec:keyboard_enter}).

Назначение кнопок на данном уровне меню показано ниже.
\begin{center}
	\begin{tabular}{p{2cm} p{15cm}}
		\textbf{[*]} 			& - возврат на уровень~0 меню; \tabularnewline
		\textbf{[ENT]} 			& - переход к вводу значения параметра; \tabularnewline
		\textbf{[CE/C]} 		& - переход на один уровень меню вверх (возврат на уровень~1). \tabularnewline				
	\end{tabular}
\end{center}