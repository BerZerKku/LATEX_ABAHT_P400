%%% ----------
\subsection{Управление}	\label{ssec:control}

Переход к пункту меню <<Управление>> из уровня~0 меню: 

\textbf{[*]} $\rightarrow$ \textbf{[3]}.

Этот пункт меню позволяет сбросить свой или удаленный аппарат, запустить передатчик неманипулированным ВЧ сигналом и т.д. 

В зависимости от режима совместимости приемопередатчика (общий параметр <<Совместимость>>), а также количества аппаратов в линии (параметр защиты <<Тип линии>>), будут доступны различные наборы команд управления (см. Приложение~\ref{app:control}). 

Если аппаратов в линии больше двух, то в команде управления может быть добавлен номер удаленного аппарата. Например <<Сброс удаленного~1>>, означает, что будет сброшен аппарат с общим параметром <<Номер аппарата>> равным~1.

Вид индикатора данного пункта меню для АВАНТ Р400, работающего на двухконцевой линии, показан на рисунке~\ref{fig:control}. 

\begin{figure}[H]
	\centering
	
	\begin{tabular}{| m{2.5cm}  m{2.5cm} |}
		\firsthline
		330кГц	& \raggedleft I1=360мА				\tabularnewline 
		\multicolumn{2}{|l|} {0.Пуск налад.вкл.}	\tabularnewline 
		\multicolumn{2}{|l|} {1.Сброс своего. }		\tabularnewline 
		\multicolumn{2}{|l|} {2.Сброс удаленного.} 	\tabularnewline \hline
		\multicolumn{2}{ c } {$\downarrow$}			\tabularnewline
		\multicolumn{2}{|l|} {3.Пуск удаленного.} 	\tabularnewline 
		\multicolumn{2}{|l|} {4.Вызов.}				\tabularnewline 
		\lasthline
	\end{tabular} 
	
	\caption{<<Управление>> для АВАНТ Р400, работающего на двухконцевой линии}
	\label{fig:control}
\end{figure}

Назначение кнопок на данном уровне меню показано ниже.
\begin{center}
	\begin{tabular}{p{2cm} p{15cm}}
		\textbf{[0]} - \textbf{[5]} & - выбор действия; \tabularnewline 
		\textbf{[*]} 	& - возврат на уровень~0 меню; \tabularnewline
		\textbf{[ENT]} 	& - подтверждение выбранного действия; \tabularnewline
		\textbf{[CE/C]} & - переход на один уровень меню вверх (возврат на уровень~1). \tabularnewline				
	\end{tabular} 
\end{center}