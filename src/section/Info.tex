%%% ----------
\subsection{Информация}	\label{ssec:info}

Переход к пункту меню <<Информация>> из уровня~0 меню: 

\textbf{[*]} $\rightarrow$ \textbf{[8]}.

Внешний вид индикатора данного пункта меню показан на рисунке~\ref{fig:info}.
 
 \begin{figure}[H]
 	\centering
 	
	\begin{tabular}{| m{2.5cm}  m{2.5cm} |}
		\firsthline
		330кГц-1	& \raggedleft I1=167мА			\tabularnewline 
		\multicolumn{2}{|l|} {Прошивка ПИ MCU}		\tabularnewline
		\multicolumn{2}{|l|} {Значение: 07.62} 		\tabularnewline 
		\multicolumn{2}{|l|} {}						\tabularnewline 
		\lasthline
	\end{tabular} 
	
	\caption{<<Информация>>}
	\label{fig:info}
\end{figure}

Этот пункт не виден на уровне~1 меню. Используется для просмотра текущих версий прошивок аппарата (см. таблицу~\ref{tab:info}), а так же дополнительных сервисных функций.

\begin{tabularx}{\linewidth}{| M{1cm} | m{5.5cm}| X |}
	\caption{Информация}  	 
	\label{tab:info}	\tabularnewline
    
    \firsthline
    
    \centering № п/п & 
    \centering Показания индикатора &     
    \centering Описание
    \tabularnewline \hline  
    \endfirsthead
    
    \multicolumn{3}{l}{Продолжение таблицы~\ref{tab:info}} 
    \tabularnewline \hline 
    \centering № п/п & 
    \centering Показания индикатора &     
    \centering Описание
    \tabularnewline \hline 
  	\endhead
    
    \multicolumn{3}{r}{продолжение следует\ldots} 
	\endfoot
	\endlastfoot
    
    1	& Прошивка ПИ MCU	& Версия прошивки микроконтроллера на плате индикации блока БСП. \tabularnewline \hline
    2	& Прошивка БСП MCU	& Версия прошивки микроконтроллера на плате блока БСП. \tabularnewline \hline
    3	& Прошивка БСП DSP	& Версия прошивки цифрового сигнального процессора на плате блока БСП. \tabularnewline \hline
    4	& Прошивка БСЗ ПЛИС	& Версия прошивки ПЛИС на плате блока БСЗ. \tabularnewline
  
    \lasthline
\end{tabularx}

Назначение кнопок на данном уровне меню показано ниже.
\begin{center}
	\begin{tabular}{p{2cm} p{15cm}}
		\textbf{[*]} 			& - возврат на уровень~0 меню; \tabularnewline
		\textbf{[ENT]} 			& - переход к вводу значения параметра; \tabularnewline
		\textbf{[CE/C]} 		& - переход на один уровень меню вверх (возврат на уровень~1). \tabularnewline				
	\end{tabular}
\end{center}