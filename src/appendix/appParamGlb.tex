\ESKDappendix{Обязательное}{Параметры общие} \label{app:paramGlb}

\begin{tabularx}{\linewidth}{| M{4.6cm} |*{7}{p{0.3cm} |} m{5.8cm} |}
	\caption{Параметры общие}  	 
	\label{tab:appParamGlb}	\tabularnewline 
    
    \firsthline
    
    \multirow{2}{*}{Параметр} & \multicolumn{7}{c|}{Совместимость} & \centering \multirow{2}{*}{Описание} \tabularnewline \cline{2-8}
     &
    \centering \begin{sideways} АВАНТ Р400~ \end{sideways} &
    \centering \begin{sideways} ПВЗ-90 \end{sideways} &
    \centering \begin{sideways} АВЗК-80 \end{sideways} &
    \centering \begin{sideways} ПВЗЛ \end{sideways} &
    \centering \begin{sideways} ПВЗК \end{sideways} &
    \centering \begin{sideways} ПВЗУ \end{sideways} &
    \centering \begin{sideways} ПВЗ \end{sideways} & 
   	\tabularnewline \hline 
    \endfirsthead
	
	\multicolumn{9}{l}{Продолжение таблицы~\ref{tab:appParamGlb}}
	\tabularnewline \hline
    \multirow{2}{*}{Параметр} & \multicolumn{7}{c|}{Совместимость} & \centering \multirow{2}{*}{Описание} \tabularnewline \cline{2-8}
     &
    \centering \begin{sideways} АВАНТ Р400~ \end{sideways} &
    \centering \begin{sideways} ПВЗ-90 \end{sideways} &
    \centering \begin{sideways} АВЗК-80 \end{sideways} &
    \centering \begin{sideways} ПВЗЛ \end{sideways} &
    \centering \begin{sideways} ПВЗК \end{sideways} &
    \centering \begin{sideways} ПВЗУ \end{sideways} &
    \centering \begin{sideways} ПВЗ \end{sideways} & 
   	\tabularnewline \hline 
  	\endhead

    \multicolumn{11}{r}{продолжение следует\ldots}
	\endfoot
	\endlastfoot
	
	Совместимость		& $\bullet$ & $\bullet$ & $\bullet$ & $\bullet$ & $\bullet$ & $\bullet$ & $\bullet$ & Режим работы приемопередатчика, обеспечивающий совместимость с приемопередатчиками других типов. \tabularnewline \hline
	Синхронизация часов	& $\bullet$ &   &   &   & $\bullet$ &   &   & Включение/выключение синхронизации часов между приемопередатчиками. \tabularnewline \hline
	Uвых номинальное	& $\bullet$ & $\bullet$ & $\bullet$ & $\bullet$ & $\bullet$ & $\bullet$ & $\bullet$ & Номинальное выходное напряжение, за изменением которого следит приемопередатчик при включенном параметре <<Контроль вых. сигнала>>.  \tabularnewline \hline
	Сетевой адрес		& $\bullet$ & $\bullet$ & $\bullet$ & $\bullet$ & $\bullet$ & $\bullet$ & $\bullet$ & Адрес аппарата в локальной сети.  \tabularnewline \hline
	Частота				& $\bullet$ & $\bullet$ & $\bullet$ & $\bullet$ & $\bullet$ & $\bullet$ & $\bullet$ & Средняя частота номинальной полосы частот. \tabularnewline \hline
	Номер аппарата		& $\bullet$ & $\bullet$ & $\bullet$ & $\bullet$ & $\bullet$ & $\bullet$ & $\bullet$ & Порядковый номер аппарата в канале.  \tabularnewline \hline
	Контроль вых.сигнала & $\bullet$ & $\bullet$ & $\bullet$ & $\bullet$ & $\bullet$ & $\bullet$ & $\bullet$ & Включение либо отключение контроля за уровнем выходного сигнала передатчика.  \tabularnewline \hline
	Порог ПРЕДУПР по КЧ & $\bullet$ &   &   &   & $\bullet$ &   &   & Порог срабатывания предупредительной сигнализации при изменении запаса по затуханию сигнала КЧ.  \tabularnewline \hline
	Порог ПРЕДУПР по РЗ &   & $\bullet$ & $\bullet$ & $\bullet$ &   & $\bullet$ & $\bullet$ & Порог срабатывания предупредительной сигнализации при изменении запаса по затуханию сигнала РЗ.  \tabularnewline \hline
	Порог аварии по КЧ	& $\bullet$ &   &   &   & $\bullet$ &   &   & Порог срабатывания аварийной сигнализации при изменении запаса по затуханию сигнала КЧ. \tabularnewline \hline
	Коррекция напряжения & $\bullet$ & $\bullet$ & $\bullet$ & $\bullet$ & $\bullet$ & $\bullet$ & $\bullet$ & Используется при несовпадении показаний индикатора «U» на дисплее блока БСП и измеренного с помощью внешних приборов напряжения на выходе усилителя мощности. \tabularnewline \hline
	Коррекция тока 		& $\bullet$ & $\bullet$ & $\bullet$ & $\bullet$ & $\bullet$ & $\bullet$ & $\bullet$ & Используется при несовпадении показаний индикатора «I» на дисплее блока БСП и измеренного с помощью внешних приборов тока на выходе усилителя мощности.  \tabularnewline \hline
	Снижение ответа АК	&   &   &   & $\bullet$ &   &   &   & Снижение уровня второго сигнала ответа на запрос автоконтроля. \tabularnewline \hline
	Допустимые провалы	&   &   &   &   &   & $\bullet$ &   & Порог по уровню тока выхода приемника, порождаемого просечками ВЧ сигнала, при одновременном пуске передатчиков манипулированным сигналом в ходе проверки ДФЗ. \tabularnewline
	
    \lasthline
\end{tabularx} 