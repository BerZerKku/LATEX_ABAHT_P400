%%% ----------
\ESKDappendix{Обязательное}{Расшифровка сообщений в журнале событий} \label{app:journal}

\begin{tabularx}{\linewidth}{| M{1cm} | m{3.5cm}| X |}
	\caption{Записи журнала событий} 
	\label{tab:appJournal}	
	\tabularnewline 
    
    \firsthline
    
    \centering № & 
    \centering Событие &     
    \centering Описание 
    \tabularnewline \hline  
    \endfirsthead
    
    \multicolumn{3}{l}{Продолжение таблицы \ref{tab:appJournal}} 
    \tabularnewline \hline 
    \centering № & 
    \centering Событие &     
    \centering Описание
    \tabularnewline \hline 
  	\endhead
    
    \multicolumn{3}{r}{продолжение следует\ldots} 
	\endfoot
	\endlastfoot
    
    1 	& Н.чт.FLASH	& Ошибка чтения данных из микросхемы памяти FLASH на блоке БСП.	\tabularnewline \hline
    2 	& Н.зп.FLASH	& Ошибка записи данных в микросхему памяти FLASH на блоке БСП.	\tabularnewline \hline
    3 	& Н.чт.PLIS		& Ошибка чтения данных из микросхемы ПЛИС на блоке БСП.	\tabularnewline \hline
    4 	& Н.зп.PLIS		& Ошибка записи данных в микросхему ПЛИС на блоке БСП.	\tabularnewline \hline
    5 	& Н.зп.2RAM		& Ошибка записи данных в микросхему двухпортового внешнего ОЗУ на блоке БСП.	\tabularnewline \hline
    6 	& Н.чт.2RAM		& Ошибка чтения данных из микросхемы двухпортового внешнего ОЗУ на блоке БСП.	\tabularnewline \hline
    7 	& Н.раб.DSP		& Неисправность цифрового сигнального процессора на блоке БСП. 	\tabularnewline \hline
    8 	& Вост.р.DSP	& Восстановление нормальной работы цифрового сигнального процессора на блоке БСП.	\tabularnewline \hline
    
    9 	& Низк. Uвых	& При запущенном передатчике, напряжение на выходе усилителя мощности снизилось в два раза по сравнению с напряжением, указанных в параметре <<Uвых номинальное>>.	\tabularnewline \hline
	10 	& Выс. Uвых		& При запущенном передатчике, напряжение на выходе усилителя мощности выросло в полтора раза по сравнению с напряжением, указанным в параметре <<Uвых номинальное>>.	\tabularnewline \hline
	11	& Н.св. с УМ	& Неисправность микроконтроллера на измерительной плате в блоке усилителя мощности.	\tabularnewline \hline
	12	& Н. часов		& Сбой часов приемопередатчика.	\tabularnewline \hline
	13	& Нет бл.БСЗ	& Блок БСЗ отсутствует в каркасе с блоками, либо неисправен.	\tabularnewline \hline
	14	& Н.верс.БСЗ	& Версия блока БСЗ не соответсвтует текущей версии приемопередатчика, либо блок БСЗ неисправен.	\tabularnewline \hline
	15	& Н.пер. БСЗ	& Положение переключателей S1.1 \ldots S1.4 на блоке БСЗ не соотвествует значению параметра <<Тип защиты>>.	\tabularnewline \hline
	16	& Нет с. МАН	& На входах <<Ман1>> или <<Ман2>> отсутствует напряжение манипуляции в течении времени установленного в параметре <<Допустимое время без МАН>>.	\tabularnewline \hline
	17	& Перезапуск	& Включение электропитания приемопередатчика.	\tabularnewline \hline
	18	& Изм.режима	& Изменение режима работы приемопередатчика.	\tabularnewline \hline
	19	& Ош.записи		& Неисправность выходной цепи приемника: <<ПРМ~2>>, либо <<РЗ вых>>.	\tabularnewline \hline
	20	& Ош.контр.		& Неисправность выходной цепи приемника: <<ПРМ~2>>, либо <<РЗ вых>>.	\tabularnewline \hline
	21	& Сниж.у.ПРМ	& Снижение запаса по затуханию.	\tabularnewline \hline
	22	& Нет с. ПРМ	& Удаленный приемопередатчик не отвечает на вызов автоконтроля.	\tabularnewline \hline
	23	& Нет с.ПУСК	& Неисправность входной цепи <<Пуск>>.	\tabularnewline \hline
	24	& Нет с.СТОП	& Неисправность	входной цепи <<Стоп>>.\tabularnewline \hline
	25	& Выключение	& Выключение электропитания приемопередатчика.	\tabularnewline \hline
	26	& Помеха в п	& При автоконтроле, при незапущенных своем и удаленном передатчиках, обнаружен сигнал на входе приемника.	\tabularnewline \hline
	27	& Неиспр.ДФЗ	& Во время автоконтроля, в тесте ДФЗ обнаружена неисправность.	\tabularnewline \hline
	28	& Уд: Нет АК	& Удаленный приемопередатчик не получил ответа при автоконтроле.	\tabularnewline \hline
	29	& Уд: Помеха	& Удаленный приемопередатчик обнаружил помеху при автоконтроле.	\tabularnewline \hline
	30	& Уд: Н. ДФЗ	& Удаленный приемопередатчик обнаружил неисправность в тесте ДФЗ при автоконтроле.	\tabularnewline \hline
	31	& Уд: Н. ВЫХ	& Удаленный приемопередатчик обнаружил неисправность выходной цепи приемника.	\tabularnewline 
	
	\lasthline
\end{tabularx} 