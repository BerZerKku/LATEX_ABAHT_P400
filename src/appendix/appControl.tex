%%% ----------
\ESKDappendix{Обязательное}{Управление} \label{app:control}

\begin{tabularx}{\linewidth}{| M{0.6cm} | m{4.9cm} | m{11.2cm} |}
	\caption{Команды управления в совместимости Р400 }  	 
	\label{tab:appСontrol_p400}	\tabularnewline
    
    \firsthline
    
    \centering № п/п &
    \centering Показания индикатора &
    \centering Измеряемый параметр
    \tabularnewline \hline
    \endfirsthead

    \multicolumn{3}{l}{Продолжение таблицы~\ref{tab:appControl_p400}}
    \tabularnewline \hline
    \centering № п/п &
    \centering Показания индикатора &
    \centering Измеряемый параметр
    \tabularnewline \hline
  	\endhead

    \multicolumn{3}{r}{продолжение следует\ldots}
	\endfoot
	\endlastfoot
    
    \multicolumn{3}{|c|}{2-х концевая линия} \tabularnewline \hline
    0	& Пуск налад.вкл. \newline Пуск налад.выкл.	& Наладочный пуск передатчика: включение/выключение передатчика на пять минут. \tabularnewline \hline
    1	& Сброс своего. 		& Сброс неисправностей приемопередатчика.	 			\tabularnewline \hline
    2	& Сброс  удаленного. 	& Сброс неисправностей на удаленном приемопередатчике. 	\tabularnewline \hline
    3	& Пуск удаленного.		& Пуск удаленного передатчика на 20 с.  				\tabularnewline \hline
    4	& Вызов.				& Включение вызывного сигнала на удаленном приемопередатчике (приглашение к переговорам). \tabularnewline \hline

    \multicolumn{3}{|c|}{3-х концевая линия} \tabularnewline \hline
    0	& Пуск налад.вкл. \newline Пуск налад.выкл.	& Наладочный пуск передатчика: включение/выключение передатчика на пять минут. \tabularnewline \hline
    1	& Сброс своего. 		& Сброс неисправностей приемопередатчика.	 				\tabularnewline \hline
    2	& Сброс удаленного X. 	& Сброс неисправностей на удаленном приемопередатчике X. 	\tabularnewline \hline
    3	& Сброс удаленного Y. 	& Сброс неисправностей на удаленном приемопередатчике Y. 	\tabularnewline \hline
    4	& Пуск удаленного X.	& Пуск удаленного передатчика X на 20~с.  					\tabularnewline \hline
    5	& Пуск удаленного Y.	& Пуск удаленного передатчика Y на 20~с.  					\tabularnewline \hline
    6	& Пуск удаленных.		& Пуск всех удаленных передатчиков на 20~c.  				\tabularnewline \hline
    7	& Вызов.				& Включение вызывного сигнала на удаленном приемопередатчике (приглашение к переговорам). \tabularnewline
  
    \lasthline
\end{tabularx} 


\begin{tabularx}{\linewidth}{| M{0.6cm} | m{4.9cm} | m{11.2cm} |}
	\caption{Команды управления в совместимости ПВЗ-90}  	 
	\label{tab:appControl_pvz90}	\tabularnewline
    
    \firsthline
    
    \centering № п/п & 
    \centering Показания индикатора &     
    \centering Измеряемый параметр
    \tabularnewline \hline  
    \endfirsthead
    
    \multicolumn{3}{l}{Продолжение таблицы~\ref{tab:appControl_pvz90}} 
    \tabularnewline \hline 
    \centering № п/п & 
    \centering Показания индикатора &     
    \centering Измеряемый параметр
    \tabularnewline \hline 
  	\endhead
    
    \multicolumn{3}{r}{продолжение следует\ldots} 
	\endfoot
	\endlastfoot
    
    0	& Пуск налад.вкл. \newline Пуск налад.выкл.	& Наладочный пуск передатчика: включение/выключение передатчика на пять минут. \tabularnewline \hline
    1	& Сброс своего. 		& Сброс неисправностей приемопередатчика.	 			\tabularnewline \hline
    2	& Сброс  удаленного. 	& Сброс неисправностей на удаленном приемопередатчике. 	\tabularnewline \hline
	% TODO в Р400м и конфигураторе сигнала Вызов нет
    3	& Вызов.				& Включение вызывного сигнала на удаленном приемопередатчике (приглашение к переговорам). \tabularnewline
  
    \lasthline
\end{tabularx}


\begin{tabularx}{\linewidth}{| M{0.6cm} | m{4.9cm} | m{11.2cm} |}
	\caption{Команды управления в совместимости АВЗК-80}  	 
	\label{tab:appControl_avzk80}	\tabularnewline
    
    \firsthline
    
    \centering № п/п & 
    \centering Показания индикатора &     
    \centering Измеряемый параметр
    \tabularnewline \hline  
    \endfirsthead
    
    \multicolumn{3}{l}{Продолжение таблицы~\ref{tab:appControl_avzk80}} 
    \tabularnewline \hline 
    \centering № п/п & 
    \centering Показания индикатора &     
    \centering Измеряемый параметр
    \tabularnewline \hline 
  	\endhead
    
    \multicolumn{3}{r}{продолжение следует\ldots} 
	\endfoot
	\endlastfoot
    
    0	& Пуск налад.вкл. \newline Пуск налад.выкл.	& Наладочный пуск передатчика: включение/выключение передатчика на пять минут. \tabularnewline \hline
    1	& Сброс своего. 		& Сброс неисправностей приемопередатчика.	 			\tabularnewline \hline
	% TODO в Р400м и конфигураторе сигнала Вызов нет
    2	& Вызов.				& Включение вызывного сигнала на удаленном приемопередатчике (приглашение к переговорам). \tabularnewline
  
    \lasthline
\end{tabularx}


\begin{tabularx}{\linewidth}{| M{0.6cm} | m{4.9cm} | m{11.2cm} |}
	\caption{Команды управления в совместимости ПВЗУ-Е }  	 
	\label{tab:appControl_pvzue}	\tabularnewline
    
    \firsthline
    
    \centering № п/п & 
    \centering Показания индикатора &     
    \centering Измеряемый параметр
    \tabularnewline \hline  
    \endfirsthead
    
    \multicolumn{3}{l}{Продолжение таблицы~\ref{tab:appControl_pvzue}}
    \tabularnewline \hline 
    \centering № п/п & 
    \centering Показания индикатора &     
    \centering Измеряемый параметр
    \tabularnewline \hline 
  	\endhead
    
    \multicolumn{3}{r}{продолжение следует\ldots} 
	\endfoot
	\endlastfoot
    
    \multicolumn{3}{|c|}{2-х концевая линия} \tabularnewline \hline
    0	& Пуск налад.вкл. \newline Пуск налад.выкл.	& Наладочный пуск передатчика: включение/выключение передатчика на пять минут. \tabularnewline \hline
    1	& Сброс своего. 		& Сброс неисправностей приемопередатчика.	 			\tabularnewline \hline
    2 	& Пуск удаленного.		& Пуск удаленного передатчика на 15 с.  				\tabularnewline \hline 
    3 	& Пуск удален. МАН		& Пуск удаленного передатчика манипулированным сигналом на 15 с.		\tabularnewline \hline 
    4 	& Пуск удал-ых. МАН		& Пуск всех удаленных передатчиков манипулированным сигна-лом на 15 с.	\tabularnewline \hline 
    5	& Вызов.				& Включение вызывного сигнала на удаленном приемопередатчике (приглашение к переговорам). \tabularnewline \hline
    
    \multicolumn{3}{|c|}{3-х концевая линия} \tabularnewline \hline
    0	& Пуск налад.вкл. \newline Пуск налад.выкл.	& Наладочный пуск передатчика: включение/выключение передатчика на пять минут. \tabularnewline \hline
    1	& Сброс своего. 		& Сброс неисправностей приемопередатчика.	 				\tabularnewline \hline
    2	& Сброс  удаленного X. 	& Сброс неисправностей на удаленном приемопередатчике X. 	\tabularnewline \hline
    3	& Сброс  удаленного Y. 	& Сброс неисправностей на удаленном приемопередатчике Y. 	\tabularnewline \hline
    4	& Пуск удал. МАН X.		& Пуск удаленного передатчика X манипулированным сигналом на 15 с.  	\tabularnewline \hline
    5	& Пуск удал. МАН Y.		& Пуск удаленного передатчика Y манипулированным сигналом на 15 с.  	\tabularnewline \hline
    6 	& Пуск удал-ых. МАН		& Пуск всех удаленных передатчиков манипулированным сигна-лом на 15 с.	\tabularnewline \hline 
    7	& Вызов.				& Включение вызывного сигнала на удаленном приемопередатчике (приглашение к переговорам). \tabularnewline \hline
    
    \multicolumn{3}{|c|}{4-х концевая (и более) линия} \tabularnewline \hline
    0	& Пуск налад.вкл. \newline Пуск налад.выкл.	& Наладочный пуск передатчика: включение/выключение передатчика на пять минут. \tabularnewline \hline
    1	& Сброс своего. 		& Сброс неисправностей приемопередатчика.	 				\tabularnewline \hline
    2	& Сброс  удаленного X. 	& Сброс неисправностей на удаленном приемопередатчике X. 	\tabularnewline \hline
    3	& Сброс  удаленного Y. 	& Сброс неисправностей на удаленном приемопередатчике Y. 	\tabularnewline \hline
    4	& Сброс  удаленного Z. 	& Сброс неисправностей на удаленном приемопередатчике Z. 	\tabularnewline \hline
    5	& Пуск удал. МАН X.		& Пуск удаленного передатчика X манипулированным сигналом на 15 с.  	\tabularnewline \hline
    6	& Пуск удал. МАН Y.		& Пуск удаленного передатчика Y манипулированным сигналом на 15 с.  	\tabularnewline \hline
    7	& Пуск удал. МАН Z.		& Пуск удаленного передатчика Z манипулированным сигналом на 15 с.  	\tabularnewline \hline
    8 	& Пуск удал-ых. МАН		& Пуск всех удаленных передатчиков манипулированным сигна-лом на 15 с.	\tabularnewline \hline 
    9	& Вызов.				& Включение вызывного сигнала на удаленном приемопередатчике (приглашение к переговорам). \tabularnewline
  
    \lasthline
\end{tabularx}


\begin{tabularx}{\linewidth}{| M{0.6cm} | m{4.9cm} | m{11.2cm} |}
	\caption{Команды управления в совместимости ПВЗЛ}  	 
	\label{tab:appControl_pvzl}	\tabularnewline
    
    \firsthline
    
    \centering № п/п & 
    \centering Показания индикатора &     
    \centering Измеряемый параметр
    \tabularnewline \hline  
    \endfirsthead
    
    \multicolumn{3}{l}{Продолжение таблицы~\ref{tab:appControl_pvzl}} 
    \tabularnewline \hline 
    \centering № п/п & 
    \centering Показания индикатора &     
    \centering Измеряемый параметр
    \tabularnewline \hline 
  	\endhead
    
    \multicolumn{3}{r}{продолжение следует\ldots} 
	\endfoot
	\endlastfoot
    
    0	& Пуск налад.вкл. \newline Пуск налад.выкл.	& Наладочный пуск передатчика: включение/выключение передатчика на пять минут. \tabularnewline \hline
    1	& Сброс своего. 		& Сброс неисправностей приемопередатчика.	 						\tabularnewline \hline
    2	& Пуск АК удаленный 	& Внеочередной запуск автоконтроля на удаленном приемопередатчике. 	\tabularnewline \hline
    3 	& Пуск ПРД				& Пуск удаленного передатчика на 10 секунд.							\tabularnewline \hline
    4	& Вызов.				& Включение вызывного сигнала на удаленном приемопередатчике (приглашение к переговорам). \tabularnewline
  
    \lasthline
\end{tabularx}


\begin{tabularx}{\linewidth}{| M{0.6cm} | m{4.9cm} | m{11.2cm} |}
	\caption{Команды управления в совместимости Линия-Р }  	 
	\label{tab:appControl_liner}	\tabularnewline
    
    \firsthline
    
    \centering № п/п &
    \centering Показания индикатора &
    \centering Измеряемый параметр
    \tabularnewline \hline
    \endfirsthead

    \multicolumn{3}{l}{Продолжение таблицы~\ref{tab:appControl_liner}}
    \tabularnewline \hline
    \centering № п/п &
    \centering Показания индикатора &
    \centering Измеряемый параметр
    \tabularnewline \hline
  	\endhead

    \multicolumn{3}{r}{продолжение следует\ldots}
	\endfoot
	\endlastfoot

    \multicolumn{3}{|c|}{2-х концевая линия} \tabularnewline \hline
    0	& Пуск налад.вкл. \newline Пуск налад.выкл.	& Наладочный пуск передатчика: включение/выключение передатчика на пять минут. \tabularnewline \hline
    1	& Сброс своего. 		& Сброс неисправностей приемопередатчика.	 			\tabularnewline \hline
    2	& Сброс  удаленного. 	& Сброс неисправностей на удаленном приемопередатчике. 	\tabularnewline \hline
    3	& Пуск удаленного.		& Пуск удаленного передатчика на 20 с.  				\tabularnewline \hline
    4	& Вызов.				& Включение вызывного сигнала на удаленном приемопередатчике (приглашение к переговорам). \tabularnewline \hline

    \multicolumn{3}{|c|}{3-х концевая линия} \tabularnewline \hline
    0	& Пуск налад.вкл. \newline Пуск налад.выкл.	& Наладочный пуск передатчика: включение/выключение передатчика на пять минут. \tabularnewline \hline
    1	& Сброс своего. 		& Сброс неисправностей приемопередатчика.	 				\tabularnewline \hline
    2	& Сброс удаленного X. 	& Сброс неисправностей на удаленном приемопередатчике X. 	\tabularnewline \hline
    3	& Сброс удаленного Y. 	& Сброс неисправностей на удаленном приемопередатчике Y. 	\tabularnewline \hline
    4	& Пуск удаленного X.	& Пуск удаленного передатчика X на 20~с.  					\tabularnewline \hline
    5	& Пуск удаленного Y.	& Пуск удаленного передатчика Y на 20~с.  					\tabularnewline \hline
    6	& Пуск удаленных.		& Пуск всех удаленных передатчиков на 20~c.  				\tabularnewline \hline
    7	& Вызов.				& Включение вызывного сигнала на удаленном приемопередатчике (приглашение к переговорам). \tabularnewline
  
    \lasthline
\end{tabularx} 


\begin{tabularx}{\linewidth}{| M{0.6cm} | m{4.9cm} | m{11.2cm} |}
	\caption{Команды управления в совместимости ПЗВК }  	 
	\label{tab:appControl_pvzk}	\tabularnewline
    
    \firsthline
    
    \centering № п/п &
    \centering Показания индикатора &
    \centering Измеряемый параметр
    \tabularnewline \hline
    \endfirsthead

    \multicolumn{3}{l}{Продолжение таблицы~\ref{tab:appControl_pvzk}}
    \tabularnewline \hline
    \centering № п/п & 
    \centering Показания индикатора &     
    \centering Измеряемый параметр
    \tabularnewline \hline 
  	\endhead

    \multicolumn{3}{r}{продолжение следует\ldots}
	\endfoot
	\endlastfoot
    
    0	& Пуск налад.вкл. \newline Пуск налад.выкл.	& Наладочный пуск передатчика: включение/выключение передатчика на пять минут. \tabularnewline \hline
    1	& Сброс своего. 		& Сброс неисправностей приемопередатчика.	 			\tabularnewline \hline
    % TODO в конфигураторе только сигнал Сброс своего
    2	& Сброс  удаленного. 	& Сброс неисправностей на удаленном приемопередатчике. 	\tabularnewline \hline
    3	& Пуск удаленного.		& Пуск удаленного передатчика на 20 с.  				\tabularnewline \hline
    4	& Вызов.				& Включение вызывного сигнала на удаленном приемопередатчике (приглашение к переговорам). \tabularnewline 
  
    \lasthline
\end{tabularx} 


\begin{tabularx}{\linewidth}{| M{0.6cm} | m{4.9cm} | m{11.2cm} |}
	\caption{Команды управления в совместимости ПВЗУ}  	 
	\label{tab:appControl_pvzu}	\tabularnewline
    
    \firsthline
    
    \centering № п/п & 
    \centering Показания индикатора &     
    \centering Измеряемый параметр
    \tabularnewline \hline  
    \endfirsthead
    
    \multicolumn{3}{l}{Продолжение таблицы~\ref{tab:appControl_pvzu}} 
    \tabularnewline \hline 
    \centering № п/п & 
    \centering Показания индикатора &     
    \centering Измеряемый параметр
    \tabularnewline \hline 
  	\endhead
    
    \multicolumn{3}{r}{продолжение следует\ldots} 
	\endfoot
	\endlastfoot
    
    0	& Пуск налад.вкл. \newline Пуск налад.выкл.	& Наладочный пуск передатчика: включение/выключение передатчика на пять минут. \tabularnewline \hline
    1	& Сброс своего. 		& Сброс неисправностей приемопередатчика.	 			\tabularnewline \hline
    2	& Вызов.				& Включение вызывного сигнала на удаленном приемопередатчике (приглашение к переговорам). \tabularnewline
  
    \lasthline
\end{tabularx}


\begin{tabularx}{\linewidth}{| M{0.6cm} | m{4.9cm} | m{11.2cm} |}
	\caption{Команды управления в совместимости ПВЗ}  	 
	\label{tab:appControl_pvz}	\tabularnewline
    
    \firsthline
    
    \centering № п/п & 
    \centering Показания индикатора &     
    \centering Измеряемый параметр
    \tabularnewline \hline  
    \endfirsthead
    
    \multicolumn{3}{l}{Продолжение таблицы~\ref{tab:appControl_pvz}} 
    \tabularnewline \hline 
    \centering № п/п & 
    \centering Показания индикатора &     
    \centering Измеряемый параметр
    \tabularnewline \hline 
  	\endhead
    
    \multicolumn{3}{r}{продолжение следует\ldots} 
	\endfoot
	\endlastfoot
    
    0	& Пуск налад.вкл. \newline Пуск налад.выкл.	& Наладочный пуск передатчика: включение/выключение передатчика на пять минут. \tabularnewline \hline
    1	& Сброс своего. 		& Сброс неисправностей приемопередатчика.	 			\tabularnewline \hline
  
    \lasthline
\end{tabularx}

