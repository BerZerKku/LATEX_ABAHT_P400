%%% ----------
\ESKDappendix{Обязательное}{Неисправности и предупреждения} \label{app:err}

\begin{tabularx}{\linewidth}{| M{2cm} | m{4.5cm}| X |}
	\caption{Общие неисправности} 
	\label{tab:appError_glb_error}	
	\tabularnewline 
    
    \firsthline
    
    \centering Код & 
    \centering Показания индикатора &     
    \centering Описание неисправности 
    \tabularnewline \hline  
    \endfirsthead
    
    \multicolumn{3}{l}{Продолжение таблицы \ref{tab:appError_glb_error}} 
    \tabularnewline \hline 
    \centering Код & 
    \centering Показания индикатора &     
    \centering Описание неисправности 
    \tabularnewline \hline 
  	\endhead
    
    \multicolumn{3}{r}{продолжение следует\ldots} 
	\endfoot
	\endlastfoot
    
    0x0001 & Неиспр.чт.FLASH	& Неисправность при чтении данных из микросхемы FLASH-памяти на блоке БСП.	\tabularnewline \hline
    0x0002 & Неиспр.зап.FLASH	& Неисправность при записи данных в микросхему FLASH-памяти на блоке БСП. 	\tabularnewline \hline
    0x0004 & Неиспр.чт.PLIS		& Неисправность при чтении данных из микросхемы ПЛИС на блоке БСП.	\tabularnewline \hline
    0x0008 & Неиспр.зап.PLIS	& Неисправность при записи данных в микросхему ПЛСИ на блоке БСП.	\tabularnewline \hline
    0x0010 & Неиспр.зап.2RAM	& Неисправность при записи данных в микросхему двухпортового внешнего ОЗУ на блоке БСП	\tabularnewline \hline
    0x0020 & АК-нет ответа N	& Удаленный приемопередатчик не отвечает на вызов автоконтроля. N - номер не ответившего приемопередатчика. \tabularnewline \hline
    0x0040 & АК-Снижен.запаса	& Снижение запаса по затуханию. \tabularnewline \hline
    0x0080 & Помеха в линии		& При автоконтроле, при незапущенных своем и удаленном приемопередатчиках, обнаружен сигнал на выходе приемника - помеха в линии. \tabularnewline \hline
    0x0100 & Неиспр.DSP			& Неисправность цифрового сигнального процессора на блоке БСП. \tabularnewline \hline
    0x0200 & Неиспр.чт.2RAM		& Неисправность при чтении данных из микросхемы двухпортового внешнего ОЗУ на блоке БСП. \tabularnewline \hline
    0x0400 & Ток покоя			& Во время автоконтроля, при незапущенных своем и удаленном передатчиках, обнаружен сигнал на выходе приемника.	\tabularnewline \hline
    0x0800 & Низкое напр.вых.	& При запущенном передатчике, напряжение на выходе усилителя мощности снизилось в два раза по сравнению с напряжением, указанным в параметре <<Uвых номинальное>>.	\tabularnewline \hline
    0x1000 & Высокое напр.вых.	& При запущенном передатчике, напряжение на выходе усилителя мощности выросло в полтора раза по сравнению с напряжением, указанным в параметре <<Uвых номинальное>>.	\tabularnewline \hline
    0x2000 & Неиспр. МК УМ		& Неисправность микроконтроллера на измерительной плате в блоке усилителя мощности.	\tabularnewline \hline
    0x4000 & ВЧ тракт восст.	& Восстановление канала связи между приемопередатчиками, при установленном режиме <<АК односторонний>>. 	\tabularnewline 
    
    \lasthline
\end{tabularx} 

\begin{tabularx}{\linewidth}{| M{2cm} | m{4.5cm}| m{10cm}|}
	\caption{Общие предупреждения} 
	\label{tab:appError_glb_warning}	
	\tabularnewline 
    
    \firsthline
    
    \centering Код & 
    \centering Показания индикатора &     
    \centering Описание предупреждения 
    \tabularnewline \hline  
    \endfirsthead
    
    \multicolumn{3}{l}{Продолжение таблицы \ref{tab:appError_glb_error}} 
    \tabularnewline \hline 
    \centering Код & 
    \centering Показания индикатора &     
    \centering Описание предупреждения  
    \tabularnewline \hline 
  	\endhead
    
    \multicolumn{3}{r}{продолжение следует\ldots} 
	\endfoot
	\endlastfoot
    
     0x0001 & Установите часы	& Сбой часов приемопередатчика.	\tabularnewline 
     
    \lasthline
\end{tabularx} 

\begin{tabularx}{\linewidth}{| M{2cm} | m{4.5cm}| X|}
	\caption{Неисправности защиты}  	
	\label{tab:appError_def_error}	\tabularnewline
    
     \firsthline
    
    \centering Код & 
    \centering Показания индикатора &     
    \centering Описание неисправности 
    \tabularnewline \hline  
    \endfirsthead
    
    \multicolumn{3}{l}{Продолжение таблицы \ref{tab:appError_def_error}} 
    \tabularnewline \hline 
    \centering Код & 
    \centering Показания индикатора &     
    \centering Описание неисправности 
    \tabularnewline \hline 
  	\endhead
    
    \multicolumn{3}{r}{продолжение следует\ldots} 
	\endfoot
	\endlastfoot
    
    0x0001 & Нет блока БСЗ		& Блок БСЗ отсутствует в каркасе с блоками, либо неисправен. \tabularnewline \hline
    0x0002 & Неиспр.верс.БСЗ	& Версия блока БСЗ не соответствует текущей версии приемопередатчика, либо блок БСЗ неисправен. \tabularnewline \hline
    0x0004 & Неиспр.перекл.		& Положение переключателей S1.1 \ldots S1.4 на блоке БСЗ не соответсвует значению параметра <<Тип защиты>>. 	\tabularnewline \hline
    0x0008 & Дальний			& Неиспрвность приемопередатчика противоположного конца канала связи.	\tabularnewline \hline
    0x0010 & АК-Нет ответа N	& Удаленный приемопередатчик не отвечает на вызов автоконтроля. N - номер не ответившего приемопередатчика. \tabularnewline \hline
    0x0020 & Низкий ур. РЗ		& \tabularnewline \hline
    0x0040 & Неиспр.уд.ДФЗ N	& Удаленный приемопередатчик обнаружил неисправность в тесте ДФЗ при автоконтроле. N - номер приемопередатчика обнаружившего неисправность. \tabularnewline \hline
    0x0080 & Неиспр.уд.ВЫХ N	& Удаленный приемопередатчик обнаружил неисправность выходной цепи приемника. N - номер приемопередатчика обнаружившего неисправность. \tabularnewline \hline
    0x0100 & Неиспр.вход.ПУСК	& Неисправность входной цепи <<Пуск>>.	\tabularnewline \hline
    0x0200 & Неиспр.вход.СТОП	& Неисправность входной цепи <<Стоп>>.	\tabularnewline \hline
    0x0400 & Удал.без отв. N	& Удаленный приемопередатчик не получил ответ при автоконтроле. N - номер приемопередатчика обнаружившего неисправность.\tabularnewline \hline
    0x0800 & Неиспр.цепь ВЫХ	& Неисправность выходной цепи приемника:<<ПРМ~2>> либо <<РЗ~вых>>. \tabularnewline \hline
    0x1000 & Удал.обн.пом. N	& Удаленный приемопередатчик обнаружил помеху при автоконтроле. N - номер приемопередатчика обнаружившего неисправность. \tabularnewline \hline
    0x2000 & Неиспр.зап.ВЫХ		& Неисправность выходной цепи приемника:<<ПРМ~2>> либо <<РЗ~вых>>. \tabularnewline \hline
    0x4000 & Помеха в линии		& Во время автоконтроля, при незапущенных своем и удаленном передатчиках обнаружен сигнал на выходе приемника - помеха в линии. \tabularnewline \hline
    0x8000 & Неиспр. ДФЗ N		& Во время автоконтроля, в тесте ДФЗ обнаружена неисправность. N - номер приемопередатчика обнаружившего неисправность. \tabularnewline 
    
    \lasthline
\end{tabularx} 


\begin{tabularx}{\linewidth}{| M{2cm} | m{4.5cm}| X|}
	\caption{Предупреждения защиты}  	
	\label{tab:appError_def_warning}	\tabularnewline
    
     \firsthline
    
    \centering Код & 
    \centering Показания индикатора &     
    \centering Описание предупреждения
    \tabularnewline \hline  
    \endfirsthead
    
    \multicolumn{3}{l}{Продолжение таблицы \ref{tab:appError_def_warning}} 
    \tabularnewline \hline 
    \centering Код & 
    \centering Показания индикатора &     
    \centering Описание предупреждения 
    \tabularnewline \hline 
  	\endhead
    
    \multicolumn{3}{r}{продолжение следует\ldots} 
	\endfoot
	\endlastfoot
    
    0x0001 & АК-Сн.запаса N		& Снижение запаса по затуханию. N - номер приемопередатчика, со стороны которого увеличилось затухание. \tabularnewline \hline
    0x0002 & Нет сигнала МАН	& На входах <<Ман1>> или <<Ман2>> отсутствует напряжение манипуляции в течение времени, установленного в параметре <<Допустимое время без МАН>>. \tabularnewline \hline
    0x0004 & Порог по помехе	& По выходу приемника были накоплены импульсы помехи, суммарная длительность которых превысила значение параметра <<Порог по помехе>>. \tabularnewline \hline
    0x0008 & Автоконтроль		& В совместимости с ПВЗЛ: зафиксирован пропуск очередного автоматического пуска автоконтроля на противоположном конце линии. \newline В совместимости с ПВЗ-90: зафиксировано 12 вызовов автоконтроля от удаленного приемопередатчика, при этом свой приемопередатчик автоконтроль не проводил. \tabularnewline \hline
    0x0010 & Помеха				& Помеха в канале связи. \tabularnewline \hline
    0x0020 & Часы				& Неисправность часов приемопередатчика. \tabularnewline \hline
      
    \lasthline
\end{tabularx} 