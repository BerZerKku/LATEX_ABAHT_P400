%%% ----------
\ESKDappendix{Обязательное}{Параметры защиты} \label{app:paramDef}

\begin{tabularx}{\linewidth}{| M{4.6cm} |*{9}{p{0.3cm} |} m{5.8cm} |}
	\caption{Параметры защиты}  	 
	\label{tab:appParamDef}	\tabularnewline
    
    \firsthline
    
    \multirow{2}{*}{Параметр} & \multicolumn{9}{c|}{Совместимость} & \centering \multirow{2}{*}{Описание} \tabularnewline \cline{2-10}
     &
    \centering \begin{sideways} АВАНТ Р400~ \end{sideways} &
    \centering \begin{sideways} ПВЗ-90 \end{sideways} &
    \centering \begin{sideways} АВЗК-80 \end{sideways} &
    \centering \begin{sideways} ПВЗУ-Е* \end{sideways} &
    \centering \begin{sideways} ПВЗЛ \end{sideways} &
    \centering \begin{sideways} Линия-Р* \end{sideways} &
    \centering \begin{sideways} ПВЗК \end{sideways} &
    \centering \begin{sideways} ПВЗУ \end{sideways} &
    \centering \begin{sideways} ПВЗ \end{sideways} & 
   	\tabularnewline \hline 
    \endfirsthead
	
	\multicolumn{3}{l}{Продолжение таблицы~\ref{tab:appParamDef}}
	\tabularnewline \hline
    \multirow{2}{*}{Параметр} & \multicolumn{9}{c|}{Совместимость} & \centering \multirow{2}{*}{Описание} \tabularnewline \cline{2-10}
     &
    \centering \begin{sideways} АВАНТ Р400~ \end{sideways} &
    \centering \begin{sideways} ПВЗ-90 \end{sideways} &
    \centering \begin{sideways} АВЗК-80 \end{sideways} &
    \centering \begin{sideways} ПВЗУ-Е* \end{sideways} &
    \centering \begin{sideways} ПВЗЛ \end{sideways} &
    \centering \begin{sideways} Линия-Р* \end{sideways} &
    \centering \begin{sideways} ПВЗК \end{sideways} &
    \centering \begin{sideways} ПВЗУ \end{sideways} &
    \centering \begin{sideways} ПВЗ \end{sideways} & 
   	\tabularnewline \hline 
  	\endhead

    \multicolumn{11}{r}{продолжение следует\ldots} 
	\endfoot
	\endlastfoot
	
	Тип защиты 			& $\bullet$ & $\bullet$ & $\bullet$ & $\bullet$ & $\bullet$ & $\bullet$ & $\bullet$ & $\bullet$ & $\bullet$ & Выбор одного из типов защиты: ППЗ, ДФЗ, НЗ. В зависимости от данного параметра определяется логика работы приемо-передатчика. \tabularnewline \hline
	Тип Линии			& $\bullet$ & $\bullet$ & $\bullet$ & $\bullet$ &   & $\bullet$ & $\bullet$ & $\bullet$ & $\bullet$ & Количество приемопередатчиков в канале. \tabularnewline \hline
	Доп.время без ман	& $\bullet$ & $\bullet$ & $\bullet$ & $\bullet$ & $\bullet$ & $\bullet$ & $\bullet$ & $\bullet$ & $\bullet$ & Параметр определяет время срабатывания предупредительной сигнализации при отсутствии сигнала манипуляции на соответствующем входе приемопередатчика. \tabularnewline \hline
	Загр чувствит по РЗ & $\bullet$ & $\bullet$ & $\bullet$ & $\bullet$ & $\bullet$ & $\bullet$ & $\bullet$ & $\bullet$ & $\bullet$ & Программное загрубление чувствительности приемника сигналов защиты. \tabularnewline \hline
	Снижение уровня АК  & $\bullet$ &   &   &   &   & $\bullet$ & $\bullet$ &   &   & Снижение уровня передаваемых при автоконтроле сигналов на 6 дБ. \tabularnewline \hline
	Частота ПРД			&   & $\bullet$ & $\bullet$ & $\bullet$ & $\bullet$ &   &   & $\bullet$ & $\bullet$ & Сдвиг частоты передатчика от центра номинальной полосы для обеспечения передачи и приема на разнесенных частотах. \tabularnewline \hline
	Частота ПРМ			&   & $\bullet$ & $\bullet$ & $\bullet$ & $\bullet$ &   &   & $\bullet$ & $\bullet$ & Сдвиг частоты приемника от центра номинальной полосы для обеспечения передачи и приема на разнесенных частотах. \tabularnewline \hline
	Сдвиг пер.фронта ПРД & $\bullet$ & $\bullet$ & $\bullet$ & $\bullet$ & $\bullet$ & $\bullet$ & $\bullet$ & $\bullet$ & $\bullet$ & Задержка срабатывания выхода приемника от пуска собственного передатчика. \tabularnewline \hline
	Сдвиг зад.фронта ПРД & $\bullet$ & $\bullet$ & $\bullet$ & $\bullet$ & $\bullet$ & $\bullet$ & $\bullet$ & $\bullet$ & $\bullet$ & Задержка выключения выхода приемника по окончанию пуска собственного передатчика. \tabularnewline \hline
	Сдвиг ПРМ			& $\bullet$ & $\bullet$ & $\bullet$ & $\bullet$ & $\bullet$ & $\bullet$ & $\bullet$ & $\bullet$ & $\bullet$ & Дополнительная задержка, вводимая в тракт приемника сигнала. \tabularnewline \hline		
	Сдвиг ВЧ ПРД от ПУСК & $\bullet$ & $\bullet$ & $\bullet$ & $\bullet$ & $\bullet$ & $\bullet$ & $\bullet$ & $\bullet$ & $\bullet$ & Задержка начала передачи ВЧ передатчиком ВЧ сигнала в канал от сигнала пуск или манипуляция. \tabularnewline 

    \lasthline
\end{tabularx}

* - совместимости доступны только в спец. исполнении.
